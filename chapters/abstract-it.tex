\chapter*{\centering {\emph{Introduzione in lingua italiana}}}
In linea con il regolamento della Facoltà di Ingegneria di Modena, ed essendo il seguente elaborato scritto in lingua inglese, la seguente introduzione in lingua italiana sarà una sintesi dell'intero elaborato e sarà l'unica parte scritta in questa lingua. Per una descrizione più esaustiva si faccia riferimento al testo in inglese.\\
L'obbiettivo di questa tesi è quello di effettuare una panoramica nel campo della crittografia, capirne l'importanza e i campi di applicazione presentando anche alcuni esempi di algoritmi e schemi crittografici moderni. Si conclude infine con la presentazione di un protocollo crittografico, il protocollo HTTPS, oggi ampiamente utilizzato e di notevole importanza.
\\
%introduzione
Viene fatta inizialmente un'introduzione per sottolineare l'importanza della crittografia nel passato, come ad esempio durante la seconda guerra mondiale, e nel presente, con l'avvento dei computer e di internet. Si passa poi a fare una panoramica generale sulla crittografia, introducendo in parte il gergo utilizzato e analizzando i concetti di confidenzialità, integrità e autenticazione. La confidenzialità è la caratteristica di uno schema di generare testi cifrati che, a una entità esterna che non possiede la chiave segreta, non diano alcuna informazione riguardante il messaggio originale. Nella pratica questa definizione viene rilassata e considerata valida solo per avversari efficienti, i quali possono anche avere una probabilità trascurabile, ma non nulla, di ottenere informazioni il messaggio cifrato. Questo rilassamento viene fatto perché il tempo necessario per rompere lo schema è sufficientemente elevato da considerare l'attacco infattibile.\\
 Integrità e autenticazione sono invece due concetti la cui distinzione è abbastanza sfumata e a volte anche messa in discussione. Per entrambi si utilizzano le funzioni di hash, ovvero funzioni deterministiche e unidirezionali. Queste funzioni ricevono in input il messaggio che si vuole spedire e restituiscono una stringa di lunghezza fissata. Dipendentemente dal contesto, l'output di queste funzioni è chiamato \emph{hash del messaggio} oppure \emph{checksum}. Semplificando, il mittente di un messaggio calcola l'hash di questo e lo spedisce insieme al messaggio. Il ricevente ricalcolerà lui stesso l'hash del messaggio ricevuto e lo confronta con l'hash ricevuto per confermarne l'integrità.
\\
%schemi a chiave privata
Nel successivo capitolo vengono presentati i primi schemi crittografici: gli schemi a chiave privata (o simmetrici). Questi schemi hanno la particolarità di utilizzare una singola chiave segreta, sia durante la fase di cifratura, sia durante la fase di decodifica. Per questo motivo la chiave sarà confidenziale esclusivamente tra le due parti coinvolte nella comunicazione. Esistono diversi metodi per costruire schemi a chiave privata, uno di questi è attraverso l'uso di reti a sostituzione e permutazione. Per l'integrità e l'autenticazione, negli schemi a chiave privata si fa uso dei message authentication codes (MAC), costruiti utilizzando le sopracitate funzioni di hash. Vengono infine mostrate le costruzione di uno schema a chiave privata, AES, e di un MAC, HMAC.
\\
%schemi a chiave pubblica
Nel capitolo 4 vengono presentati gli schemi a chiave pubblica (o asimmetrici). Il capitolo comincia con una sezione dedicata a una introduzione riguardate la matematica modulare e la teoria dei gruppi, entrambi alla base della costruzione degli schemi a chiave pubblica. Successivamente si introduce il problema di RSA e la costruzione dell'omonimo schema asimmetrico in una delle sue forme più semplici. Infinite viene mostrata la controparte dei MAC negli schemi a chiave privata, ovvero le firme digitali, utilizzate per associare a una chiave pubblica l'identità di una persona o una qualsiasi entità.
\\
%https
Infine nel quinto capitolo, viene presentato un protocollo che fa uso sia di schemi a chiave pubblica, sia schemi a chiave privata. I primi sono utilizzati per instaurare la comunicazione e scambiarsi la chiave segreta utilizzata poi con lo schema a chiave privata. Questa combinazione di schemi, detta \emph{ibrida}, viene utilizzata perché gli schemi simmetrici sono molto più efficienti della loro controparte. Il protocollo presentato è HTTPS ed è oggi il principale protocollo usato per la comunicazione tra web browsers e web servers. Si è scelto questo protocollo perché, tra i vari schemi usati, sono spesso presenti sia RSA sia AES. Inoltre la verifica dell'autenticità di un sito web, avviene proprio attraverso le firme digitali, attraverso l'uso di varie infrastrutture dette certificates authorities.

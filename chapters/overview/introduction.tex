Cryptography is the study of using digital coding to secure data accessed. In other words, to ensure that data can only be access by authorized entities.\\
To define cryptography jargon, we introduce a situation where an entity Alice wants to send a message to another entity Bob through a communication channel. Cryptography is relevant when there's an adversary that tries to access the data sent through the channel without legitimate authorization.\\
The \emph{plaintext} is the message that Alice wants to send to Bob. The \emph{ciphertext} is the data that goes through the channel and one of the resources that an adversary can access.
The process of converting the plaintext to the ciphertext is called \emph{encryption}, while the process to convert the ciphertext to the plaintext is called \emph{decryption}.
Encryption and decryption are defined by the \emph{cryptographic scheme}, a set of algorithms that Alice and Bob decide to use before the actual communication, and one or more \emph{keys}, that can be confidential and shared only between the authorized entities or they can also be public depending on the type of scheme used.\\
Regardless of the scheme's type used, which will be discussed later in detail, there are three main features that a cryptographic scheme should have to be defined secure: \emph{confidentiality}, \emph{integrity}, and \emph{authentication}.

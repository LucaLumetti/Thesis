Cryptography is the study of using digital coding to secure data access. In other words to ensure that data can only be access by authorized entities.\\
To describe the problem that cryptography is trying to address we will use a specific jargon to represent entities: Alice and Bob.\\
Alice has a message that wants to share with Bob so they are both authorized to read the message. Cryptography becomes relevant when there's an adversary, or an attacker, that try to access the data sent from Alice without the legitimate authorization.\\
The \emph{plaintext} is the message that Alice wants to send to Bob. The \emph{ciphertext} and is the data that goes through the channel and one of the resources that an adversary can access.
The process of converting the plaintext to the ciphertext is called \emph{encryption} while the process to convert the ciphertext to the plaintext is called \emph{decryption}.
Encryption and decryption are defined by the \emph{cryptographic scheme}, a set of algorithms that Alice and Bob decide to use before the actual communication, and one or more \emph{keys}, that can be confidential only between the two entities or they can also be public depending on the type of scheme used.\\
Regardless of the type of scheme used, which will be discussed later in detail, there are three main features that a cryptographic scheme should have to be defined secure: \emph{confidentiality}, \emph{integrity} and \emph{authentication}.
% In the symmetric cryptography, Alice and Bob share a private key known only by them used to by the encryption and decryption algorithms, while in the  asymmetric cryptography, Alice and Bob both have a pair of key, one public and  one private. The public key of Bob is used by Alice to encrypt the messages and the private key is used by Bob to decrypt the message received by Alice

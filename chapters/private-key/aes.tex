\section{Advanced Encryption Standard}
AES is a common symmetric key block cipher used worldwide to protect data. It is the successor of \emph{DES} (Data Encryption Standard), another block cipher that now has been classified insecure because of its key length, and after some attacks became too efficient. It has been chosen after a public competition held to meet the need for a new encryption standard.\\
Before introducing the concrete algorithm, we must introduce Galois field, a finite field used during a step of AES.

\subsection{Galois Field}
A Galois Field (or finite field) is a field with a finite number of elements. The most common fields used are given by the integers $\mathsf{mod}\,p$ where $p$ is a prime number, also written as $\mathsf{GF}[p]$. For $n > 1$, with $\mathsf{GF}[p^n]$ we refer to all polynomials of degree $n-1$ with coefficients coming from $\mathsf{GF}[p]$. As an example, $\mathsf{GF}[2^3]$ is a field with $8$ elements (the integers from 0 to 7) and they can all be represented as a polynomial of degree $2$ (6 can be represented as $110_2$ or $x^2+x$).\\
Even if the sum between polynomials is trivial, the multiplication works differently: an irreducible polynomial $g(x)$ of degree $n$ is chosen, then the multiplication in $\mathsf{GF}[p^n]$ is the ordinary product, except you have to take the remainder of the division by $g(x)$. A different $g(x)$ defines a different field.

\subsection{Encryption Method}
The security of AES comes from confusion and diffusion achieved by a substitution\=/permutation network with a block size of 128 bit and it supports three different key lengths: 128, 198, and 256 bits. Every key length uses a different number of rounds, i.e., 10, 12, and 14, respectively.\\
Each block of length 128 bits is rearranged in a 4x4 bytes array, called \emph{state}, and for every round the following steps are executed:
\begin{enumerate}
    \item{\textbf{AddRoundKey:} A 16 byte round key is derived from the master key and it's interpreted as a 4x4 array. Then, the key is XORed with the state array.\\This step consists of computing $a_{i,j} = a_{i,j} \oplus k_{i,j}$ for every $i,j \in {1,...,4}$ where $a_{i,j}$ is the $i^{th}$ row and $j^{th}$ column of the state array and $k_{i,j}$ is the $i^{th}$ row and $j^{th}$ column of the key array.}
    \item{\textbf{SubBytes:} Each byte of the state array is substituted by another byte, according to a single fixed lookup table \emph{S}. So, it consists in computing $a_{i,j} = S(a_{i,j})$.}
    \item{\textbf{ShiftRows:} Each row of the state array is cyclically shifted to the left as follows: the first row of the array is untouched, the second row is shifted one place to the left, the third row is shifted two places to the left, and the fourth row is shifted three places to the left.}
    \item{\textbf{MixColumns:} In this step, each column is mixed via an invertible linear transformation. Specifically, each column is interpreted as a polynomial over $\mathsf{GF}[2^8]$ with $g(x) = x^4 + 1$, and is multiplied with a fixed polynomial $c(x) = 3x^3 + x^2 + x + 2$.}
\end{enumerate}
In the final round, the MixColumns stage is replaced with an additional AddRoundKey step.

\section{Message Authentication Codes}\label{sec:mac}
Everything we have seen until now assures only confidentiality. As introduced in \autoref{chap:overview}, message authentication codes (MAC) also called \emph{tags} are used to introduce both integrity and authentication.\\
Even if an encryption scheme assures confidentiality, integrity and authentication are important to assure that the message is not tampered by an external entity. This is a propriety of encryption schemes called \emph{malleability}.\\
For example, in CBC an attacker, because the IV and ciphertext blocks are directly XORed with the next block output, a bit changed in the IV or the ciphertext block corresponds with a bit changed in the plaintext. If the attacker is able to guess the format of the uncrypted message the attacker could be able to change sensitive informations of the plaintext.\\
A message authentication code (MAC) is a tuple of PPT algorithms $(\mathsf{Gen},\mathsf{Mac},\mathsf{Vrfy})$ such that:
\begin{itemize}
    \item{$\mathsf{Gen(\cdot)}$: it takes as input $n$ and outputs an uniformly distrbuted key of length $n$.\\We will write this as $k \leftarrow \mathsf{Gen}(1^n)$.}
    \item{$\mathsf{Mac(\cdot)}$: It receive as input a key $k \in \{0,1\}^n$ and a message $m \in \{0,1\}^{*}$ to output a \emph{tag} $t \in \{0,1\}^{*}$.\\We will write this as $t \leftarrow \mathsf{Mac}_k(m)$.}
    \item{$\mathsf{Vrfy(\cdot,\cdot)}$ on input $k \in \{0,1\}^n$, $m \in \{0,1\}^{*}$ and $t \in \{0,1\}^{*}$ it output a bit $b \in {0,1}$.\\We will write this as $b \leftarrow \mathsf{Vrfy}_k(m, t)$.}
\end{itemize}

\section{Message Authentication Codes}\label{sec:mac}
Everything we have seen until now assures only confidentiality. As introduced in \autoref{chap:overview}, message authentication codes (MAC), also called \emph{tags}, are used to introduce both integrity and authentication.\\
Even if an encryption scheme assures only confidentiality, integrity and authentication are important to assure that the message is not tampered by an external entity. This is a property of encryption schemes called \emph{malleability}.\\
For example, in CBC an attacker, because the IV and ciphertext blocks are directly XORed with the next block output, a bit changed in the IV or the ciphertext block corresponds with a bit changed in the plaintext. If the attacker can guess the format of the unencrypted message, the attacker could be able to change sensitive information on the plaintext.\\
\begin{nopbreak}
A message authentication code (MAC) is a tuple of PPT algorithms $(\mathsf{Gen},\mathsf{Mac},\mathsf{Vrfy})$ such that:
\begin{itemize}
    \item{$\mathsf{Gen(\cdot)}$: it takes as input $n$ and outputs a uniformly distributed key of length $n$.\\We will write this as $k \leftarrow \mathsf{Gen}(1^n)$.}
    \item{$\mathsf{Mac_k(\cdot)}$: It receive as input a key $k \in \{0,1\}^n$ and a message $m \in \{0,1\}^{*}$ to output a \emph{tag} $t \in \{0,1\}^{*}$.\\We will write this as $t \leftarrow \mathsf{Mac}_k(m)$.}
    \item{$\mathsf{Vrfy_k(\cdot,\cdot)}$ on input $k \in \{0,1\}^n$, $m \in \{0,1\}^{*}$ and $t \in \{0,1\}^{*}$ it outputs a bit $b \in {0,1}$.\\We will write this as $b \leftarrow \mathsf{Vrfy}_k(m, t)$.}
\end{itemize}
It also required that for every $n$, every $k$, and every $m$ it holds that:
$$
\mathsf{Vrfy}_k(m, \mathsf{Mac}_k(m)) = 1
$$
\end{nopbreak}
The security of MAC is that an adversary can't forge a valid tag for a new message in a reasonable time.\\
The construction of MAC can be based on block ciphers, like \emph{CBC-MAC}, or collision-resistant hash functions built with the Merkle-Damg\r{a}rd transform (see \autoref{sec:collisionresistant} and \autoref{sec:merkledamgard}), like \emph{NMAC} or \emph{HMAC}.

\subsection{HMAC construction}
With $H^s_{\mathsf{IV}}(x)$ the hash function constructed with Merkle-Darmg\r{a}rd transform with $z_0$ set to an arbitrary value $\mathsf{IV}$ and we also define a keyed version of the compression function $h^s(x)$ used in $\mathsf{H}$ by $h^s_k(x) = h^s(k\;||\;x)$. Two constants are also defined: $\mathsf{opad}$ and $\mathsf{ipad}$ of length $n$ (the length of a single block of the input to $\mathsf{H}$).\\
The string $\mathsf{opad}$ is formed by repeating the byte $\mathsf{0x5C}$ as many times needed and the string $\mathsf{ipad}$ is formed in the same way using the byte $\mathsf{0x36}$.\\
The HMAC construction is the same defined above in this section, with two additions:
\begin{itemize}
    \item{The $\mathsf{Gen}$ algorithm also run the key generation for the hash function obtaining the value $s$.}
    \item{The $\mathsf{Mac}_k$ algorithm is computed by:
$$
    \mathsf{HMAC}^s_k(x) = \mathsf{H}^s_\mathsf{IV}(k \oplus \mathsf{opad}\;||\;\mathsf{H}_\mathsf{IV}(k \oplus \mathsf{ipad}\;||\;x))
$$
        }
\end{itemize}
The hash function $\mathsf{H}$ can be any cryptographic hash function like SHA-1, SHA-256, etc..., and the relative HMACs are named HMAC-SHA1, HMAC-SHA256. The cryptographic strength of HMAC depends on the properties of the underlying hash function, so using hash functions like MD5 or SHA-1 is not recomended.
%encrypt then auth
\subsection{Chosen-Ciphertext Secure Encryption}
By using Messages Authentication Codes with Block Ciphers, we are now able to build an encryption scheme that is secure against chosen-ciphertext attacks. To achieve this, the encryption scheme will have the property that the adversary is unable to forge ciphertext that was not generated by the legitimate users, so the decryption oracle, that the adversary can use, become useless.\\
The following definition is the join of a CPA-secure encryption scheme $(\mathsf{Gen}_E,\mathsf{Enc},\mathsf{Dec})$ and a secure message authentication code $(\mathsf{Gen}_M, \mathsf{Mac}, \mathsf{Vrfy})$:
\begin{itemize}
    \item{$\mathsf{Gen}'(\cdot)$: upon input $n$, choose $k_1 \leftarrow \mathsf{Gen}_E(n)$ and $k_2 \leftarrow \mathsf{Gen}_M(n)$.}
    \item{$\mathsf{Enc}'_k(\cdot)$: upon input key ($k_1$, $k_2$) and a message $m$, output the pair $(\mathsf{Enc}_{k_1}(m), \mathsf{Mac}_{k_2}(c))$.}
    \item{$\mathsf{Dec}'_k(\cdot, \cdot)$: upon input key ($k_1$, $k_2$) and the pair ($c$, $t$), where $c$ is the ciphertext and $t$ is the MAC tag, if $\mathsf{Vrfy}_{k_2}(c, t) = 1$, then output $\mathsf{Dec}_{k_1}(c)$, else output $\mathsf{null}$.}
\end{itemize}

\section{TLS}
The TLS protocol allows client/server applications to communicate, while preventing the tampering and eavesdropping of information. In the HTTPS protocol, only the server authenticates to the client but not vice versa. Is the evolution of another protocol called SSL, today marked as insecure and not anymore supported by browsers. Even the first versions of TLS are planned to be deprecated during 2020. The most recent version of TLS is 1.3.\\
To establish a secure connection, the client and the server, before transmitting any other information, perform a \emph{handshake}.
\subsection{Handshake}
TLS handshake occurs after a TCP connection has been opened via a TCP handshake. The last ACK sent by the client also contains the first step of the TLS handshake.
\subsubsection{1 - Client Hello}
This is the first step performed by the client and is also know as cryptographic negotiation. The client shares with the server the list of his supported TLS versions, his cipher suite, and it might send options about the ciphers or other client information. The client also generates a random 32-byte number, used later to generate symmetric keys, and a session ID used to identify the connection.\\
Example of Client Hello taken using tshark:
\begin{lstlisting}
Handshake Protocol: Client Hello
Length: 510
Version: TLS 1.2 (0x0303)
Random: 55 2a 89 9e f4 21 04 49 f3 17 6a 39 8b cc 4c 39 ab 44 24 3b 25 ce 1b 95 cc 9a 47 52 ac 1c 29 18
Session ID Length: 32
Session ID: 54 33 56 39 b8 5f 27 4d 56 cb b1 0f 45 6a b2 92 e9 8d a2 95 97 bd f9 8d f6 b3 b2 a0 4e 9c 4e 7f
Cipher Suites Length: 32
Cipher Suite: TLS_AES_128_GCM_SHA256
Cipher Suite: TLS_AES_256_GCM_SHA384
... more ciphers ...
Cipher Suite: TLS_RSA_WITH_AES_128_GCM_SHA256
Cipher Suite: TLS_RSA_WITH_AES_256_CBC_SHA256
Extensions Length: 405
... extensions ...
\end{lstlisting}

\subsubsection{2 - Server Hello}
The server replies to the client hello with a server hello. The server sends a message containing the TLS server version and a cipher suite, both chosen among the ones received from the Client Hello. It also sends a 32-byte random number and the session ID received by the client.\\
Example of Server Hello taken using tshark:
\begin{lstlisting}
Handshake Protocol: Server Hello
Handshake Type: Server Hello (2)
Length: 96
Version: TLS 1.2 (0x0303)
Random: 72 f2 70 6b 47 f1 d5 98 73 20 68 78 7c 26 a3 7d da 54 d8 30 3a 48 8c bf a7 90 68 95 c5 c0 68 97
Session ID Length: 32
Session ID: 54 33 56 39 b8 5f 27 4d 56 cb b1 0f 45 6a b2 92 e9 8d a2 95 97 bd f9 8d f6 b3 b2 a0 4e 9c 4e 7f
Cipher Suite: TLS_ECDHE_RSA_WITH_CHACHA20_POLY1305_SHA256 (0xcca8)
Extensions Length: 24
... extensions ...
\end{lstlisting}

\subsubsection{3 - Server Certificate}
\par
The server sends his certificate and his public key to the client to prove his identity.\\
\begin{lstlisting}
Handshake Protocol: Certificate
Handshake Type: Certificate (11)
Length: 2568
Certificates Length: 1174
Certificates (1174 bytes)
Certificate Length: 1174
Certificate: 308204923082... (id-at-commonName=Let's Encrypt Authority X3, id-at-organizationName=Let's Encrypt, id-at-countryName=US)
version: v3 (2)
serialNumber: 0x0a0141420000015385736a0b85eca708
Algorithm Id: 1.2.840.113549.1.1.11 (sha256WithRSAEncryption)
modulus: 00 9c d3 0c f0 5a e5 2e 47 b7 72 5d 37 83 b3 68 63 30 ea d7 35 26 19 25...
publicExponent: 65537
... more ...
\end{lstlisting}

\subsubsection{4 - Client Key Exchange}
\par
A pre-master secret key is created by the client and sent to the server. How the key is created might depend on the cipher suite selected. The key is encrypted using the server public key. Both client and server compute the \emph{master secret} key using a pseudorandom function that takes, as input, the pre-master secret and the 32-byte random value exchanged early. This master key is 48 bytes long and is used to symmetrically encrypt data with one of the private-key ciphers chosen from the cipher suite.

\subsubsection{5 - Client Handshake Finished}
\par
The client is now ready to switch to a secure environment. From now on, every data sent to the server will be encrypted using the symmetric scheme chosen and the master key. The client sends his first encrypted message, saying that the handshake for itself is finished.

\subsubsection{6 - Server Handshake Finished}
\par
Also, the server is ready to switch to a secure environment and from now on, every data sent by the server will be encrypted using the same algorithms used by the client. It sends an encrypted message, saying that the handshake for itself is terminated.

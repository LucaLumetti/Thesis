\section{HTTP}
The Hypertext Transfer Protocol is an application-layer protocol for transmitting hypermedia documents, such as HTML documents. It's often based over the TCP/IP layer and it is stateless. It works in a request/reply fashion, based on the exchange of individual messages. Requests are the messages sent by the client (usually a web browser), responses are messages sent back by the server.
\subsection{Request}
An HTTP request includes:
\begin{itemize}
    \item{\textbf{Method:} specifies the operation that the client wants to execute on the server, like GET, POST, PUT, ...}
    \item{\textbf{URL:} is the identifier of the requested object}
    \item{\textbf{Version:} the HTTP version used}
    \item{\textbf{Headers:} additional information that can be used by the server, like date, the browser used, cookies. They are not mandatory.}
\end{itemize}
Here is an example of an HTTP request:
\begin{lstlisting}
    GET /directory/page.html HTTP/1.1
    Connection: close
    User-agent: Mozilla/5.0 (X11; Linux x86_64)
    Accept: text/html, image/jpeg
    Accept-language: it-IT,en-US
\end{lstlisting}

\subsection{Response}
An HTTP response includes, besides the content of the resource requested, a header with the HTTP version, a status code, and some additional response information.
Here is an example of an HTTP response:
\begin{lstlisting}
    HTTP/1.1 200 OK
    Content-language: it
    Content-length: 18844
    Content-type: text/html; charset=UTF-8
    Date: Mon, 22 Jun 2020 21:50:53 GMT
    Server: nginx

    <!DOCTYPE html><head> ... the page ... </html>
\end{lstlisting}
\subsection{HTTPS vs HTTP}
HTTPS is an extension of HTTP, which means that the request and response format is exactly the same but the messages exchanged are encrypted by a cryptographic protocol. The protocol used is Transport Layer Security (TLS) and is the successor of Secure Socket Layer (SSL), today deprecated. That's why HTTPS is also referred to as HTTP over TLS.\\
Other differences are that HTTPS uses the well-known port 443, while HTTP uses the well-known port 80 and the URL starts with \texttt{https://} instead of \texttt{http://}.

\chapter{Introduction}\label{chap:introduction}
\par
Today people are able to access and share information with others in a way that was not even imaginable back to some decades ago. The rise of the Internet has changed how we live, how we work, and how we communicate. This was also more clear at the start of this decade because of coronavirus epidemic that forced many of us to limit our social life, use the Internet to work from home, and keep in contact with friends. The main reason why the Internet joined our life so much is that we can have a reasonable level of privacy over it thanks to tools that invisibly work under the hood of every message we send and websites we visit. This is not the only environment where cryptography is used. One of the earliest uses dates back two thousand years ago, when Julius Caesar developed a way to send secret military messages to his commanders. Cryptography also had a main role in World War I and mostly in World War II, where each nation had his own cipher to send military information to allies. The most famous was the Enigma machine used by the Germany Reich to encrypt the messages. After years of work, the English mathematician Alan Turing was able to decode the messages sent by germans and bring a huge advantage to Allies. Historians have estimated that this fact reduced the duration of the war by two years. Since then and with the evolution and spread of computers, cryptography had to adapt to the enormous computational power we have today and be accessible to everyone and not just military and national agencies. In this paper is explained which cryptographic tools are used, how they are constructed, and which type of security they give to conclude with the introduction of a commonly used protocol named HTTPS.


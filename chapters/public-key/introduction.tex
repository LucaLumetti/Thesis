Until now we have seen how to achieve a secure communication  over an insecure channel, but not discussed yet how keys are shared and managed. In fact these are one of the main problems of symmetric schemes, specially in open system like on the Internet. For a group of $n$ entity where everyone wants to communicate with each others, the total number of secret keys that need to be generated is $\binom{n}{2} \approx \mathsf{O}(n^2)$ and these need to be distributed over a secure channel that is not always present.
Even if partial solutions where built to overcome these problem, they were not sufficient.
The first step to fully solve these problems was made in 1976 by Whitfield Diffie and Martin Hellman by publishing a paper called "New Directions in Cryptography". With this paper they laid the foundation for asymmetrics schemes. These schemes use 2 different keys called \emph{public} and \emph{private} key, the first one is used to encrypt the message while the second one to decrypt the ciphertext.
So public key scheme is a tuple $(\mathsf{Gen}, \mathsf{Enc}, \mathsf{Dec})$ such that:
\begin{itemize}
    \item{$\mathsf{Gen}(\cdot)$: takes as input the security parameter $n$ and outputs a pair of key $(pk, sk)$, respectively the public and the secret (private) key.}
    \item{$\mathsf{Enc}(\cdot)$: takes as input a public key $pk$ and a message $m$ to output the ciphertext $c \leftarrow \mathsf{Enc}_{pk}(m)$.}
    \item{$\mathsf{Dec}(\cdot)$: takes as input a secret key $sk$ and a ciphertext $c$ to output the message $m := \mathsf{Dec}_{sk}(c)$.}
\end{itemize}
To establish a communication between 2 entities Alice and Bob, first the key pairs $(A_{pk}, A_{sk})$ and $(B_{pk}, B_{sk})$ are generated, then both public keys are shared among the entities and used to used to encrypt messages. Says tha Alice wants to send a message $m$ to Bob, Alice craft the ciphertext using Bob's public key $c \leftarrow  \mathsf{Enc}_{B_{pk}}(m)$ then send it to Bob that will decode it with his secret key $m = \mathsf{Dec}_{B_{sk}}(c)$.\\
In the scenario where $n$ entities participate in the communication, the number of keys involved is $2n$ and every public key can be freely distributed throught insecure channels.\\

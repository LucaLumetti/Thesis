\section{Crittografia simmetrica}
Uno schema crittografico a chiave simmetrica è stato il primo metodo crittografico inventato.
In questo caso se si vuole scambiare un messaggio tra due parti A e B, quest'ultime devono essere entrambe in possesso della stessa chiave che verrà utilizzata da entrambe le parti sia per crittografare il messaggio, sia per decodificarlo.
Uno schema di questo tipo è quindi definito da una tripla $(\mathsf{Gen},\mathsf{Enc},\mathsf{Dec})$ dove:
\begin{itemize}
    \item{\textbf{$\mathsf{Gen}(\cdot)$:} è un algoritmo randomizzato che a fronte di un input $1^{n}$ genera una chiave $k$ tale che $|k| \geq n$. E' quindi definito nel seguente modo: $k \leftarrow \mathsf{Gen}(1^{n})$}.
    \item{\textbf{$\mathsf{Enc}(\cdot)$:} è un algoritmo polinomiale che puo' essere di tipo deterministico o probabilistico. Ha come input una chiave $k$ e un messaggio $m$ e restituisce un messaggio cifrato $c$. Potendo essere un algoritmo probabilistico, scriveremo: $c \leftarrow \mathsf{Enc}_k(m)$.}
    \item{\textbf{$\mathsf{Dec}(\cdot)$:} è un algoritmo polinomiale deterministico che a ha come input una chiave $k$ e un messaggio cifrato $c$ e restituisce un messaggio $m$. Essendo un algoritmo deterministico scriveremo: $m := \mathsf{Dec}_k(c)$.}
\end{itemize}
Inoltre, per ogni $n$ e per ogni $k$ generato da $\mathsf{Gen}(1^{n})$ vale la seguente relazione:
$$
    m = \mathsf{Dec}_k(\mathsf{Enc}_k(m))\,,\;\;\;\;\forall m \in \{0, 1\}^{*}
$$

\section{Crittografia asimmetrica}
Uno schema crittografico a chiave simmetrica (anche detto a chiave pubblica) si differenzia dal precendente in quanto entrambe le parti in una comunicazione possiedono una coppia di chiavi distinte $(pk, sk)$ che chiameremo rispettivamente \emph{chiave pubblica} e \emph{chiave privata}. Riprendendo l'esempio di una comunicazione tra due parti A e B, in questo caso all'inizio della comunicazione le due parti si scambiano le rispettive chiavi pubbliche $A_{pk}$ e $B_{pk}$. Se A vuole mandare un messaggio a B, utilizzerà $B_{pk}$ per crittografare il messaggio prima di spedirlo. Successivamente B per decifrare il messaggio userà la sua chiave privata $B_{sk}$.\\
Questo rappresenta un'importante rivoluzione in quanto la comunicazione tra le due parti avviene senza la necessità di precedere la comunicazione con lo scambiarsi o accordarsi riguardo informazioni private. \\
La definizione di schema crittografico a chiave asimmetrica è quindi definito da una tupla di algoritmi $(\mathsf{Gen}, \mathsf{Enc}, \mathsf{Dec})$ tali che:
\begin{itemize}
    \item{\textbf{$\mathsf{Gen}(\cdot)$:} è l'algoritmo che riceve come input $1^n$ e restituisce come output una coppia di chiavi $(pk, sk)$ che rispettivamente chiameremo \emph{chiave pubblica} e \emph{chiave privata}. Per convenienza assumeremo che $|pk| \geq n$ e $|sk| \geq n$ e che n può essere ottenuto a partire dalla coppia di chiavi.
    \item{\textbf{$\mathsf{Enc}(\cdot)$:} è un algoritmo che ha come input una chiave pubblica $pk$ e un messaggio $m$ e restituisce come output un messaggio cifrato $c$. E' definito come: $c \leftarrow \mathsf{Enc}_{pk}(m)$}.
    \item{\textbf{$\mathsf{Dec}(\cdot)$:} è un algoritmo cheha come input una chiave privata $sk$ e un messaggio cifrato $c$ restituendo come output il messaggio $m$. E' definito come: $m := \mathsf{Dec}_{sk}(m)$}}
\end{itemize}
Inoltre, per ogni $n$ e ogni $(pk, sk)$ restituiti da $\mathsf{Gen}(1^{n})$ vale:
$$
    m = \mathsf{Dec}_{sk}(\mathsf{Enc}_{pk}(m))\,,\;\;\;\;\forall m \in \{0, 1\}^{*}
$$
